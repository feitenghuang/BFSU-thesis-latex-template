%模板说明

%制作者:黄飞腾;制作时间:2021年5月;联系方式:huangfeiteng@bfsu.edu.cn

%本latex模板适用于北京外国语大学学位论文。需要使用者具备基础的latex知识。运行前,需要预先把参考文献制作成bib文件,并在模板最后替换文件名。
%本模板生成的论文为pdf格式,从致谢页开始,致谢页页码为 ii。生成pdf后,需要手动把封面和声明(由word模板制作,请按照要求制作电子签名)贴在前面,生成完整版论文。
%本模板仅供参考,不保证格式的完全正确,请根据要求对各项细节进行核对修改。如果使用者的毕业年份中,对学位论文格式有新的要求,请根据具体要求修改相应格式,例如字体字号等。本模板中的图,表,参考文献等仅作为样例供参考。

%本模板仅用于北外毕业生的学术论文写作,禁止用于任何商业用途。作者保留一切解释和追究的权利。


\documentclass[12pt]{article} %12pt是4号 %ctexart会报错
\usepackage[fleqn]{amsmath}
\usepackage{amssymb}
\usepackage{amsthm} %proof环境在这个包里
\usepackage{geometry}
\usepackage{array}
\usepackage{bm}
\usepackage{graphicx}
%\usepackage{setspace}
\usepackage{enumitem}
\setenumerate[1]{itemsep=0pt,partopsep=0pt,parsep=\parskip,topsep=5pt} %设置enumerate环境的行间距
\usepackage{longtable}
\usepackage{booktabs}
\usepackage{graphicx,subfigure,amssymb,amsthm,amsbsy,float,natbib}
\usepackage{epstopdf}
\usepackage{color}
\usepackage{soul}
\usepackage{longtable}
\usepackage{bm}
\usepackage{rotating}
\usepackage{subfigure}
\usepackage{algorithm}
\usepackage{algorithmic}
\usepackage{diagbox}
\usepackage[fleqn]{amsmath}
\usepackage{geometry}
\usepackage{array}
\usepackage{bm}
\usepackage{setspace}
\usepackage{enumitem}
\usepackage{longtable}
\numberwithin{equation}{section} %???section??
\newcommand{\ud}{\,\mathrm{d}}  %??

%中文格式设置
\usepackage[UTF8]{ctex}%有中文时加这一行
%\usepackage{listings}%贴代码

\usepackage[colorlinks,
linkcolor=black,
anchorcolor=black,
citecolor=black,
urlcolor=black
]{hyperref}
%超链接,参考文献加入目录,颜色为黑色

\usepackage{titletoc}
\titlecontents{section}[0pt]{\addvspace{1.35pt}\filright\bf}%
{\contentspush{第\thecontentslabel\ 章\quad}}%
{}{\titlerule*[12pt]{.}\contentspage}

\usepackage{titlesec}
%章节设置 不能改中文的第一章
\titleformat{\section}{\xiaosanhao\centering\heitiB\bfseries}{第\,\thesection\,章}{1em}{}% em表示章到标题间的空格
\titlespacing*{\section}{0pt}{0pt}{6pt} %段前,段后
\titleformat{\subsection}{\sihao\heitiB\bfseries}{\thesubsection}{1em}{}% em表示章到标题间的空格
\titlespacing*{\subsection}{0pt}{0pt}{3pt} %段前,段后,具体数值单位没懂

%自定义字体
\newcommand{\xiaosanhao}{\fontsize{15pt}{\baselineskip}\selectfont}
\newcommand{\sihao}{\fontsize{14pt}{\baselineskip}\selectfont}


\DeclareMathAlphabet{\mathsfsl}{OT1}{cmss}{m}{sl}
\newcommand{\dif}{\mathrm{d}}
\newcommand{\me}{\mathrm{e}}
\newcommand{\mi}{\mathrm{i}}
\newcommand{\mCov}{\mathrm{Cov}}
\newcommand{\mVar}{\mathrm{Var}}
\newcommand{\mVec}[1]{\boldsymbol{#1}}
\newcommand{\tensor}[1]{\mathsfsl{#1}}
\newcommand{\nnub}{\nonumber}
\newtheorem{theorem}{定理}%需要中文环境在这里改就好
\newtheorem{lemma}{引理}
\newtheorem{transform}{Transformation}
\newtheorem{corollary}{Corollary}
\newtheorem{property}{Property}
\newtheorem{proposition}{Proposition}
\newtheorem{definition}{定义}
%\newtheorem{algorithm}{Algorithm}
\newtheorem{assumption}{Assumption}
\newtheorem{conjecture}{Conjecture}
\newtheorem{guideline}{Guideline}

\renewcommand{\contentsname}{目录}
\renewcommand{\figurename}{图}
\renewcommand{\tablename}{表}
\renewcommand\refname{参考文献}




%中文字体设置,
%\usepackage{xeCJK}   %中文字体
%\setCJKfamilyfont{kaitibold}{Kaiti SC Bold}
%\newcommand{\kaitiB}{\CJKfamily{kaitibold}}   %楷体加粗
%\setCJKfamilyfont{heitibold}{Source Han Sans CN}
\newcommand{\heitiB}{\CJKfamily{heitibold}}   %黑体加粗
%\setCJKfamilyfont{songtibold}{Songti SC}
%\newcommand{\songtiB}{\CJKfamily{songtibold}}   %宋体加粗

%\usepackage{ctexcap} %加了会报错


\title{制造商面对多个竞争零售商时的预售融资决策}
\author{黄飞腾}
\date{} %2020.11

%不包含封面和原创声明

\geometry{a4paper,left=3cm,right=2.5cm,top=2.5cm,bottom=2.5cm}
\hypersetup{colorlinks=true,linkcolor=black}%链接颜色

\begin{document}
\setlength{\baselineskip}{21.06pt} %行距 %1.3*12*1.35

\begin{center}\textbf{\xiaosanhao \heitiB %小三号
		致\quad 谢
	}\
\end{center}

~\\
\indent
光阴似箭,日月如梭。

\pagenumbering{roman}
\setcounter{page}{2} %致谢页页码为 ii。生成pdf后,需要手动把封面和声明(由word模板制作)贴在前面


\clearpage

%摘要
\begin{center}\textbf{\xiaosanhao \heitiB
		摘\quad 要
	}
\end{center}

~\\
\indent

本研究探讨了一种制造型企业的预售融资策略


\noindent \textbf{关键词:}预售,资金约束,配给博弈,斯塔克伯格博弈,供应链协调 

\clearpage





\begin{center}\textbf{\xiaosanhao \heitiB
		ABSTRACT
	}
\end{center}

~\\
\indent

This research explores a pre-sale financing strategy for manufacturing companies


\noindent \textbf{KEY WORDS:}Advance Selling, Cash flow constraint , Rationing Game, Stackelberg Game, Supply Chain Coordination

\clearpage


\tableofcontents %目录
\newpage
\pagenumbering{arabic}
\setcounter{page}{1}%页码格式


\section{绪论}
\subsection{研究背景与意义}
生产型企业普遍面临融资决策和风险管理等方面的困难。

%引用:citep是带括号,cite不带
现有研究表明,零售阶段和批发阶段的预售都可以帮助企业控制风险,提高利润。目前,已有学者研究了预售在零售行业中的应用,并证明在一定的市场条件下,预售可以提升企业利润并降低风险\citep{li_advance_2013}。该类研究主要关注预售在几个方面的作用,包括预测需求,产品效用未知时的决策,划分消费者群体等。\cite{xiao_preselling_2018}研究了在一个资金短缺的制造商与一个零售商的渠道中,考虑制造商在批发阶段的预售时,制造商的生产与融资决策,并证明了预售可以帮助制造商充实现金流,降低风险,提高利润。



\subsection{研究内容与创新点}


主要的研究问题包括:
\begin{enumerate}
	\item 考虑零售市场竞争和批发阶段的预售的渠道下
	\item 第二条
\end{enumerate}




\subsection{论文框架}


\clearpage

\section{文献综述}


\subsection{供应链下游竞争的研究}



\clearpage

\section{研究模型}

%插入图片
\begin{figure}[h]
	\centering
	\caption{事件流程}\label{process} %图表的标题在上方
	\includegraphics[width=1.00\textwidth]{figure/liucheng.png}
\end{figure}

\subsection{符号说明}

%插入表格
\begin{table}[H]
	\centering
	\caption{变量说明}
\begin{tabular}{|c|c|}
	\hline
	\multicolumn{2}{|l|}{决策变量}\\\hline %左右加线写在l的两侧
	$\eta_i$ & 预售折扣率  \\
	\hline
	$q_i$ & 预售订单量  \\
	\hline
	$p_i$ & 单位零售价  \\
	\hline
	$Q$ & 生产量  \\
	\hline
	\multicolumn{2}{|l|}{内生变量}\\\hline
	$d_i$ & 市场需求量,$d=d_1+d_2$  \\
	\hline
	$m_i$ & 实际销售量  \\
	\hline
	$\pi_m,\pi_i,$ & 制造商/零售商$i$的利润,取所有利润在第二阶段的折现值  \\
	\hline
	\multicolumn{2}{|l|}{外生参数}\\\hline
	$\delta,\beta$ & 存款利率,贷款利率,满足$\beta>\delta$ \\
	\hline
	$w,c$ & 单位批发价,单位生产成本 $w>c$ \\
	\hline
	$\theta$ & 替代系数,$\theta<1$ \\
	\hline
	$\alpha_i$ & 基础市场需求 \\
	\hline
\end{tabular}
\end{table}



\clearpage


\section{分散化模型下的最优决策以及均衡解}
本章讨论各个条件下的最优决策,给出其解析表达。决策分析中,分析顺序与决策顺序相反,分析后置决策时,假设前置决策已经确定;分析前置决策时,将后置决策的最优解表示为前置决策的函数。
\subsection{制造商的产量决策$Q$}

\clearpage

\section{研究总结}

\clearpage
\section{附录}

\subsection{定理\ref{q}的证明}


综上,将零售商预售订单量的最优决策汇总为定理\ref{q}。

%\begin{thebibliography}{99}
\clearpage
\phantomsection
\addcontentsline{toc}{section}{参考文献}
\tolerance=500

\nocite{*}%添加此处到bib
\bibliographystyle{apalike}
\bibliography{myReferences}%需要预先把参考文献制作成bib文件

\end{document}
